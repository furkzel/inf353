\documentclass[20pt]{beamer}

\usetheme{metropolis}

\usefonttheme[onlymath]{serif}

\usepackage{listings}

\usepackage{multicol}


\usepackage{tikz-cd}
\usepackage{graphbox}
\usepackage{tikz}
\usepackage[tikz]{bclogo}
\usepackage{fontawesome5}
\usepackage{pgfplots}

\usepackage[all]{xy}

\usepackage{subfigure}

\usepackage{amsmath}
\usepackage{amssymb}
\usepackage{amsthm}
\usepackage{bigints}

\usepackage{diagbox}

\usepackage{geometry}
% 1920x1080 px
\geometry{
    papersize={1920pt,1080pt},
    left=25pt,
    right=25pt,
    top=25pt,
    bottom=25pt
}

\newcommand{\tikzmark}[1]{\tikz[overlay,remember picture] \node (#1) {};}
\tikzset{square arrow/.style={to path={-- ++(0,-.25) -| (\tikztotarget)}}}


\date{}

\author{İsmet Karadağ \\ Furkan Özel}
\institute{Université Galatasaray, INF 353}

\setbeamertemplate{title page}{
    \begin{center}
        \vspace{2cm}
        \usebeamerfont{title}\Huge\inserttitle\par
        \usebeamerfont{subtitle}\usebeamercolor[fg]{subtitle}\insertsubtitle\par
        \vspace{1cm}
        \usebeamerfont{author}\insertauthor\par
        \usebeamerfont{institute}\insertinstitute\par
        \vfill
        \usebeamerfont{date}\insertdate\par        
    \end{center}
}


\begin{document}

\begin{frame}[plain]
    % Insert image
    \begin{center}
        \includegraphics[width=0.3\textwidth]{7tst.png}    
    \end{center}
    \maketitle
\end{frame}

\begin{frame}{İçerik}
    \begin{multicols}{2}
        \tableofcontents
    \end{multicols}
\end{frame}

\section{Introduction}

\begin{frame}
    \frametitle{7Taste !? C'est quoi ?}
        7Taste, c'est un site web qui fournit de trouver des chansons selon les goûts de l'utilisateur. On developpe ce site web pour le projet de INF 353.
\end{frame}

\begin{frame}
    \frametitle{Plan}
    \begin{center}
        \begin{tabular}{|c|p{15cm}|p{15cm}|}
            \hline
            \textbf{Semaine} & \textbf{Tâches} & \textbf{Description} \\
            \hline
            Semaine 1 & - Déterminer le sujet du projet. & - Commencer à travailler sur le projet. \\
            \hline
            Semaine 2 & - Définir les détails de conception du projet. & - Configuration de l'authentification pour l'API Spotify. \\
            \hline
            Semaine 3 & - Création de la base de données. & - Conception de l'interface utilisateur en utilisant Bootstrap et développement du prototype. \\ & - Configuration de l'environnement .NET Core 8. \\
            \hline
            Semaine 4-7 & - Ajout de fonctionnalités à l'interface utilisateur. & - Commencer à travailler sur l'algorithme de recommandation. \\
            \hline
            Semaine 8 & - Affichage des profils d'artistes et de leurs détails. & - Configuration de la base de données MySQL et création des tables requises. \\ & - Ajout d'une fonctionnalité permettant aux utilisateurs de sauvegarder leurs chansons et listes de lecture préférées. \\
            \hline
            Semaine 9-13 & - Tester le site web et corriger les erreurs. & - Ajouter des fonctionnalités supplémentaires. \\
            \hline
            Semaine 14 & - Préparation de la présentation finale. & - Mise en ligne du site web. \\
            \hline
        \end{tabular}
    \end{center}
\end{frame}

\section{Conception}

\begin{frame}
    \frametitle{Comment ça se passe avec .Net?}

\end{frame}

\begin{frame}
    \frametitle{l'Algorithme de Recommandation}
        L'algorithme de recommandation est un domaine de recherche en informatique et en apprentissage automatique qui vise à suggérer des éléments pertinents à un utilisateur, en se basant généralement sur ses préférences et son comportement antérieur. Voici quelques informations académiques importantes concernant les algorithmes de recommandation :
        Types d'algorithmes de recommandation :
        \begin{itemize}
            \item Filtrage collaboratif : Ces algorithmes recommandent des éléments en se basant sur le comportement et les préférences d'autres utilisateurs similaires. Il existe deux types de filtrage collaboratif : basé sur l'utilisateur et basé sur l'élément.
            \item Filtrage basé sur le contenu : Ces algorithmes recommandent des éléments similaires à ceux que l'utilisateur a déjà aimés, en se basant sur des caractéristiques explicites des éléments et des préférences de l'utilisateur.
            \item Méthodes hybrides : Ces méthodes combinent plusieurs approches pour améliorer la qualité des recommandations.
        \end{itemize}
\end{frame}

\section{External Links}

\begin{frame}
    \frametitle{External Links}
    \begin{itemize}
        \item  
    \end{itemize}
\end{frame}

\end{document}